\documentclass{article}
\usepackage[utf8]{inputenc}
\usepackage{graphicx} 
\usepackage{amsmath}  
\usepackage{amssymb}  

\title{Optionsbewertung mit Hilfe der Fourier-Cosinus Methode}
\author{Denis Kocaman}
\date{30. Januar 2026}

\begin{document}
\maketitle

\section{Einleitung}
In dieser Arbeit implementiere ich die COS-Methode basierend auf dem Standardwerk von \cite{hirsaComputationalMethodsFinance2013}. Dies ist notwendig, da die charakteristische Funktion der Schlüssel zur Preisberechnung ist.

\section{Theoretischer Hintergrund}
Die charakteristische Funktion bildet die Basis für die Fourier-Inversion. Im Black-Scholes Modell ist sie definiert als der Erwartungswert des Log-Asset-Preises:

\begin{equation}
    \phi(u) = \exp\left( iu \left( \ln(S_0) + (r - \frac{1}{2}\sigma^2)T \right) - \frac{1}{2}\sigma^2 u^2 T \right)
\end{equation}

In Abbildung \ref{fig:cf_plot} sehen wir den Realteil dieser Funktion für das Black-Scholes Modell.

\begin{figure}[h]
    \centering
    \includegraphics[width=0.8\textwidth]{figures/plot_cf.pdf} 
    \caption{Visualisierung des Realteils der charakteristischen Funktion.}
    \label{fig:cf_plot}
\end{figure}

\section{Mathematische Implementierung der COS-Methode}
Der Kerngedanke der Methode ist die Approximation des Optionspreises durch eine Fourier-Cosinus-Reihe. Der Preis $V$ ergibt sich aus der Summe:

\begin{equation}
    V \approx e^{-rT} \sum_{k=0}^{N-1}{'} F_k \cdot V_k
\end{equation}

Dabei sind $F_k$ die Koeffizienten der charakteristischen Funktion und $V_k$ die Payoff-Koeffizienten. Letztere werden für eine Call-Option analytisch durch die Funktionen $\chi_k$ und $\psi_k$ bestimmt:

\begin{align}
    \chi_k(c, d) &:= \frac{1}{1 + \omega_k^2} \left[ \cos(\omega_k(d-a))e^d - \cos(\omega_k(c-a))e^c + \omega_k \sin(\omega_k(d-a))e^d - \omega_k \sin(\omega_k(c-a))e^c \right] \\
    \psi_k(c, d) &:= 
    \begin{cases} 
        d - c & \text{für } k = 0 \\
        \frac{1}{\omega_k} \left[ \sin(\omega_k(d-a)) - \sin(\omega_k(c-a)) \right] & \text{für } k > 0
    \end{cases}
\end{align}

Hierbei entspricht $\omega_k := \frac{k\pi}{b-a}$ der Frequenz der Cosinus-Wellen auf dem Integrationsintervall $[a, b]$.

% Das Literaturverzeichnis
\bibliographystyle{plain}
\bibliography{literatur}

\end{document}